\typeout{ ====================================================================}
\typeout{ this is file main.tex, created at 21-Nov-2014               }
\typeout{ maintained by Gustavo Rabello dos Anjos                             }
\typeout{ e-mail: gustavo.rabello@gmail.com                                   }
\typeout{ ====================================================================}

\documentclass[a4paper,portuges,12pt]{article}
\usepackage{babel,varioref,floatflt,wrapfig}
\usepackage{amsmath,booktabs}
\usepackage[utf8]{inputenc}
\usepackage[T1]{fontenc}   % Font Encoding T1 - caracteres acentuados.
\usepackage[top=2cm,bottom=2cm,left=2cm,right=2cm]{geometry}
\usepackage{graphicx}    % pacote para inclusao de figuras,
\usepackage{subfig}
%\pagestyle{empty}
\newcommand{\HRule}{\rule{\linewidth}{0.5mm}}

% removing reference name
\usepackage{etoolbox}
\patchcmd{\thebibliography}{\section*{\refname}}{}{}{}

%%%%%%%%%%%%%%%%%%%%%%%%%%%%%  MACROS MATEMATICOS %%%%%%%%%%%%%%%%%%%%%%%%%%%%%
\newcommand{\tr}{{\,\rm tr}\,}
\newcommand{\sen}{{\,\rm sen}\,}
\newcommand{\senh}{{\,\rm senh}\,}
\newcommand{\diverg}{{\,\rm div}\,}
\newcommand{\grad}{\,\mathbf{grad}\,}
\newcommand{\rot}{\,\mathbf{rot}\,}
\newcommand{\uvet}{\mathbf{u}}
\newcommand{\vvet}{\mathbf{v}}
\newcommand{\wvet}{\mathbf{w}}
\newcommand{\cvet}{\mathbf{c}}
\newcommand{\xvet}{\mathbf{x}}
\newcommand{\gvet}{\mathbf{g}}
\newcommand{\fvet}{\mathbf{f}}
\newcommand{\nvet}{\mathbf{n}}
\newcommand{\tvet}{\mathbf{t}}
\newcommand{\Imat}{\mathbf{I}}
\newcommand{\Eo}{\mathrm{Eo}}
\newcommand{\N}{\mathrm{N}}
\newcommand{\Mo}{\mathrm{Mo}}
%%%%%%%%%%%%%%%%%%%%%%%%%%%%%  MACROS MATEMATICOS %%%%%%%%%%%%%%%%%%%%%%%%%%%%%
										    
\begin{document}	
\typeout{ ====================================================================}
\typeout{ this is file title.tex, created at 21-Nov-2014               }
\typeout{ maintained by Gustavo Rabello dos Anjos                             }
\typeout{ e-mail: gustavo.rabello@gmail.com                                   }
\typeout{ ====================================================================}

\begin{titlepage}
\begin{center}

% Upper part of the page. The '~' is needed because \\
% only works if a paragraph has started.
\includegraphics[width=0.2\textwidth]{./figs/uerj.png}~\\[1cm]
\includegraphics[width=0.3\textwidth]{./figs/gesar.png}~\\[1cm]

\textsc{\LARGE Universidade do Estado do Rio de Janeiro}\\[1.5cm]

\textsc{\Large Relatório Final de Atividades -- CAPES}\\[0.5cm]

% Title
\HRule \\[0.4cm]
{ \huge \bfseries Sistema de Simulação Numérica de Escoamentos
Multifásicos\\[0.4cm] }
{ \Large \bfseries Projeto: A067/2013\\[0.4cm] }
\HRule \\[1.0cm]

% Author and supervisor
\noindent
\large
\emph{Bolsista Jovem Talento:}\\
Gustavo R. \textsc{Anjos}\\
\vspace{0.3cm}
\emph{Coordenador:}\\
Norberto \textsc{Mangiavacchi}\\
\vfill

% Bottom of the page
{\large \today}

\end{center}
\end{titlepage}

\typeout{ ****************** End of file title.tex ****************** }



\section{Identificação}

\noindent Dados do Coordenador: 
\begin{itemize}
	\item \textbf{Nome:} Norberto Mangiavacchi
	\item \textbf{CPF:} 732.841.227-53
	\item \textbf{Escolaridade:} Doutorado
	\item \textbf{Endereço Profissional:} Rua Fonseca Teles, 121 --
	Prédio Anexo, CEP 20940-903 São Cristóvão, RJ - Rio de Janeiro
	\item \textbf{Telefone Profissional:} +55 21 2332-4733
	\item \textbf{Endereço Eletrônico:} {\tt norberto@uerj.br}, 
	                                    {\tt norberto.mangiavacchi@gmail.com}
	\item \textbf{Página Profissional:} {\tt http://www.gesar.uerj.br}
\end{itemize}

\hspace{1cm}

\noindent Dados do Bolsista: 
\begin{itemize}
	\item \textbf{Nome:} Gustavo Rabello dos Anjos
	\item \textbf{CPF:} 042.432.587-08
	\item \textbf{Escolaridade:} Doutorado
	\item \textbf{Endereço Profissional:} Rua Fonseca Teles, 121 --
	Prédio Anexo, CEP 20940-903 São Cristóvão, RJ - Rio de Janeiro
	\item \textbf{Telefone Profissional:} +55 21 2332-4733
	\item \textbf{Endereço Eletrônico:} {\tt gustavo.anjos@uerj.br}
	\item \textbf{Página Profissional:} {\tt http://www.gesar.uerj.br}
	\item \textbf{Página Pessoal:} {\tt http://gustavo.rabello.org}
\end{itemize}

\clearpage

\section{Resumo}
Após 19 meses de projeto CAPES - Bolsa de Atração de Jovens
Talentos, os objetivos e metas previstos para este período foram
realizados com sucesso. Este documento descreve os resultados obtidos
pelo pesquisador bolsista Gustavo Rabello dos Anjos no projeto
entitulado "Sistema de Simulação Numérica de Escoamentos Multifásicos". 
\clearpage

\section{Introdução}
Deseja-se abordar o desenvolvimento e o estudo numérico de dois
problemas atuais como continuidade de projetos de pesquisa e
desenvolvimento realizados na Universidade do Estado do Rio de
Janeiro/GESAR: 

\begin{itemize}
\item Sistema de simulação numérica de produção, estocagem,
transporte e consumo de metano da decomposição de biomassa em
reservatórios de hidrelétricas (Edital FAPERJ 09/2013 – Programa de
apoio às engenharias 2013); 
\item Simulação numérica de estrutura de
não-equilíbrio em sistemas químicos, biológico e ambientais (Edital de
Cooperação Internacional – Chamada Bilateral CNPq 17/2013 – Bélgica).
\end{itemize}

Ambos com suporte técnico-científico do simulador de escoamentos
multifásicos escrito em linguagem orientada a objetos (MATLAB/C++)
utilizando moderna discretização das equações de governo através do
método de elementos finitos. Para execução do projeto proposto,
planeja-se concluir o desenvolvimento do simulador para modelos
tridimensionais e axisimétricos através da incorporação das seguintes
características: paralelização dos núcleos de cálculo intensivo em
clusters baseados em processadores de vários núcleos (multicore),
extensa validação do modelo numérico através de validações analíticas e
experimentais, este último com colaboração internacional, e publicação
dos resultados em canais de comunicação internacionais de excelência.
Com isso, objetiva-se o aperfeiçoamento das técnicas computacionais
empregadas na instituição de execução do projeto (UERJ/GESAR),
consolidando o Programa de Pós-graduação em Engenharia Mecânica na UERJ,
e a formação de recusos humanos, com aperfeiçoamento pessoal de nível
superior e pós-graduação. Os indicadores de desempenho que serão
utilizados no projeto estão baseados em publicações produzidas pelo
bolsista na UERJ, juntamente com o aluno envolvido na bolsa de IC, e
submetidos a avaliações da comunidade científica.

\section{Objetivos}
\begin{itemize}
	\item estudo e desenvolvimento de modelo em três dimensões e axisimétrico
	      para escoamentos multifásicos;
	\item estudo de escoamentos multifásicos com ondas capilares em bolhas
	      submetidas a campo de temperatura variável;
	\item realização de experimentos computacionais e laboratoriais de
		  decomposição de biomassa e produção de gases com sedimentação
		  de material orgânico; de medição da sedimentação/ressuspensão
		  e consumo de produtos;
	\item validação dos modelos desenvolvidos para sistemas existentes com
	      utilização de tecnologia de última geração;
	\item estudo e desenvolvimento de modelos tridimensionais e axisimétricos
		  para simulação numérica de estrutura de não-equilíbrio em
		  sistemas químicos, biológico e ambientais;
	\item incorporação das seguintes características ao simulador numérico:
		  paralelisação dos núcleos de cálculo intensivo em clusters
		  baseados em processadores de vários núcleos (multicore),
		  através do desenvolvimento de novos precondicionadores para a
		  aceleração dos métodos iterativos implantados e validação do
		  modelo proposto através de soluções analíticas e experimentais
		  (com participação de universidades internacionais);
	\item publicação de resultados em canais de comunicação nacionais e
	      internacionais renomados.
\end{itemize}

\section{Metodologia}
Nesta seção, detalharemos a metodologia utilizada no desenvolvimento do
simulador numérico de escoamentos multifásicos. Para tal, subdiviremos
as seções pela temática da metodogia, a saber: Equações de Governo,
Método de Elementos Finitos, Discretização do Termo de Tensão
Superficial e Solução do Sistema Linear.

\subsection{Equações de Governo}
O princípio de conservação de massa estabelece que, dado um fluido
qualquer com massa específica $\rho$ que escoa através de um volume de
controle $V$ invariante no tempo, a taxa de acumulação de massa no
interior do volume é igual ao fluxo líquido de massa para fora do
volume, em módulo. Para o problema proposto neste trabalho, onde não há
variação da massa específica do fluido em cada fase, ou seja, a massa
específica é constante em cada fase, a equação de conservação de massa
representada em coordenadas cartesianas para 3 dimensões é escrita como:

\begin{equation}
	\frac{\partial v_x}{\partial x} +
	\frac{\partial v_y}{\partial y} +
	\frac{\partial v_z}{\partial z}  
	= 
	0
	\label{eq:cm9}
\end{equation}

Para obtenção das equações de conservação da quantidade de movimento
linear para fluidos newtonianos realiza-se o mesmo procedimento de
obtenção da equação de conservação de massa. O princípio de conservação
da quantidade de movimento estabelece que, dado um fluido qualquer com
massa escoa específica $\rho$ que escoa através de um volume de controle
V, a taxa de acumulação de quantidade de movimento é igual ao fluxo
líquido de quantidade de movimento para fora do volume mais a resultante
das forças de superfície e de volume. A equação de conservação de
quantidade de movimento linear representada em coordenadas cartesianas é
dada por:

\begin{equation}
	\frac{\partial \vvet}{\partial t} + \vvet \cdot \nabla \vvet
	= 
	- \frac{1}{\rho} \nabla p +
	\nabla \cdot [\nu(\nabla \vvet + \nabla \vvet^T)] + 
	\textbf{g} + 
	\textbf{f}
\label{eq:NS3a}
\end{equation}

O princípio do transporte de energia estabelece que, dado um
fluido qualquer com massa específica $\rho$ que escoa através de um
volume de controle V, a taxa de acumulação da quantidade de energia
que entra no volume por unidade de tempo é igual ao fluxo
líquido de massa para fora do volume, na ausência de termos relacionados
ao trabalho e a dissipação toma a forma final representada em
coordeandas cartesianas como:

\begin{equation}
	\frac{\partial T}{\partial t} + \vvet \cdot \nabla T
	=  
	\nabla \cdot (k \nabla T)
\label{eq:quimica2}
\end{equation}\vspace{0.5cm}

É conveniente adimensionalizarmos as equações de conservação acima
citadas através de parâmetros conhecidos, como o diâmetro da bolha,
diâmetro do canal de escoamento dos fluidos e velocidade de escoamento.
Com isso analizamos os escoamentos através de números adimensionais que
podemos comparar com diversos problemas encontrados na literatura. As
equações de conservação em sua forma final adimensionalizada são
descritas como:

\begin{equation}
	\frac{\partial \vvet}{\partial t} + \vvet \cdot \nabla \vvet
	=
	-\frac{1}{\rho} \nabla p + \frac{1}{Re} \nabla \cdot
	[\nu ( \nabla \vvet + \nabla \vvet^T)]+
	\frac{1}{Fr^2} \mathbf{g} + 
	\frac{1}{We} \mathbf{f}
	\label{eq:final1}
\end{equation}
\begin{equation}
	\diverg \cdot \vvet = 0
	\label{eq:final2} 
\end{equation}	
\begin{equation}
	\frac{\partial T}{\partial t} + \vvet \cdot \nabla T
	=
	\frac{1}{RePr} \nabla \cdot (k \nabla T)
	\label{eq:final3}
\end{equation}\vspace{0.5cm}

\noindent onde $Fr$ representa o número de Froud, $Re$ representa o
número de Reynolds e todas as outras variáveis estão eu sua forma
adimensional.

Para solução do problema de escoamentos multifásicos, procuramos então
encontrar a solução dos campos de velocidade $\vvet$, pressão $p$ e
temperatura $T$ enunciados nos três princípios de conservação descritos
acima. Como a solução analítica destas equações são restritas a
pouquíssmimos casos práticos, onde muitas simplificações são realizadas,
utilizamos uma metodologia numérica de aproximação de solução para
encontramos os valores dos campos $\vvet$, $p$ e $T$ em geometrias
complexas que são pertinentes ao projeto proposto. 

\subsection{Método de Elementos Finitos}
Nos anos 50, o método de elementos finitos teve grande utilização na
mecânica de sólidos.  Apenas a partir da década de 70, após a
consolidação do método de Galerkin para equações de difusão,
pesquisadores começaram a investir no campo da dinâmica dos fluidos,
pode-se citar \cite{zienkiewicz1965}, \cite{oden1972}, \cite{oden1998},
\cite{chung1978}, \cite{hughes1982}, \cite{pironneau1989} e tantos
outros.  Em seguida, vários autores contribuíram para o desenvolvimento
de metodologias específicas como métodos de Petrov-Galerkin
generalizados \cite{heinrich1977}, \cite{hughes1986},
\cite{johnson1987}, métodos adaptativos \cite{oden1989}, métodos de
Taylor-Galerkin \cite{donea1984}, \cite{lohner1985}, método de Galerkin
descontínuo \cite{oden1998} etc.

O começo relativamente tardio no campo da Física de fluidos se deve,
principalmente, à presença do termo convectivo e ao forte acoplamento
entre velocidade e pressão, presentes nas equações de conservação. O
termo convectivo apresenta produto de incógnitas, caracterizando a
não-linearidade do problema e gerando operadores não simétricos, de
difícil solução.  Com o aumento do número de \emph{Reynolds}, o termo
convectivo exerce maior influência no escoamento, aumentando ainda mais
a dificuldade de solução das equações.

Outra fonte de dificuldade numérica é a condição de incompressibilidade,
que consiste em manter o campo de velocidade com divergência zero.
Então, a pressão deve ser considerada uma variável não relacionada a
qualquer equação constitutiva.  Sua presença nas equações de conservação
de quantidade de movimento tem o propósito de introduzir um grau de
liberdade a mais necessário para satisfazer a condição de divergência
zero do campo de velocidades.  Isto é, a pressão atua como um
multiplicador de Lagrange na condição de incompressibilidade resultando
em um acoplamento entre velocidade e pressão desconhecidas.

\subsection{Discretização do Termo de Tensão Superficial}

Em códigos do tipo "front-tracking" a construção da interface é feita
através de um conjunto de objetos geométricos, como triângulos,
segmentos de reta e nós, que são movidos na forma de lagrangian, onde a
malha tridimensional é fixada no espaço. Uma função adicional é necessária
para comunicação entre as malhas, uma vez que não há nenhuma
interconectividade implícita. esta abordagem mantém a interface entre as
fases com espessura zero, no qual uma representação precisa é obtida.
Apesar de sua excelente definição geométrica , as propriedades do fluido
perto da interface exigem tratamento numérico para evitar instabilidades
indesejáveis. Como consequência, estas propriedades são suavizados ao
longo da zona de transição e a espessura nula já não pode ser mais
garantida.

Ao contrário de códigos do tipo "front-tracking", a malha da interface
faz parte da malha tridimensional. Com isso não há necessidade do uso de
uma equação de transporte adicional ao modelo proposto. A malha
3-dimensional compreende um conjunto de elementos tetraédricos
distribuídos sobre o domínio e a interface é encontrado por uma função
escalar, ou seja, uma função do tipo "Heaviside", que define os nós que
pertencem a cada uma das fases mais a interface em si. Para obter uma
representação zero espessura, os nós da interface devem ser ligado de
forma consistente para que sua discretização possa ser representado por
um conjunto de triângulos interligado. Em outras palavras, cada
triângulo é uma face de dois elementos tetraédricos adjacentes. A
Figura~(\ref{fig:inter}a) mostra um diagrama esquemático da
representação da interface entre as duas fases discretas. A mesma face
do triângulo é compartilhada por dois tetraedros adjacentes, por
conseguinte, a interface de espessura zero é obtida com sucesso.

\begin{figure}[ht!]
		\subfloat[]{
		\includegraphics[scale=0.3]{figs/interface.pdf}}
		\subfloat[]{
		\includegraphics[scale=0.3]{figs/property.pdf}}
	\caption{Representação geométrica da interface entre as fases. (a) A
	interface (em cinza) é representada por um conjunto de triângulos,
	segmentos de retas e nós que fazem parte da malha de tetraedros. (b)
	(b) A propriedade do fluido $\phi$, tal como densidade ou
	viscosidade, é definida precisamente na fase $1$ e na fase $2$ com 
	espessura zero na zona de transição das fases.}
	\label{fig:inter} 
\end{figure}

Pontos negativos e positivos são encontrados nesta abordagem numérica,
mas uma característica é especialmente interessante devido à definição
das propriedades do fluido na área de transição. Do ponto de vista
macroscópico, o significado físico de uma interface é a região que
divide acentuadamente o volume ocupada por cada fase. Assim, é desejável
que uma tal uma espessura de interface deve ser mantida tão fina quanto
possível. A descrição lagrangiana garante a parte geométrica, mas,
devido à mudança brusca nas propriedades de uma fase para outra,
instabilidades numéricas podem aparecer e deteriorar a precisão da
solução. Tal problema é devido à localização da interface entre dois
elementos computacionais. Isso pode ser contornado com a formulação ALE
aliada ao método de Elementos Finitos, em que a interface não fica
localizada entre elementos de malha mas compartilha as faces de dois
elementos computacionais adjacentes e, assim, as propriedades do fluido
permanecem constante em cada elemento de malha. A transição brusca de
propriedades é, portanto, obtida com sucesso e não requer a utilização
de qualquer funções de suavização, consequentemente, garantindo precisão
no equilíbrio de forças próximas à interface.

Figura~(\ref{fig:inter}b) mostra a zona de transição entre as duas fases
coloridas por cinza claro e escuro, que foi propositadamente desenhadas
para destacar a metodologia proposta deste trabalho. Como pode ser
visto, a propriedade $\phi_1$ preenche os elementos da fase 1 e a
propriedade $\phi_2$ preenche exatamente os elementos de fase 2. Mesmo
para uma razão de propriedade alta $\phi_1 / \phi_2 =$ 1000, a
metodologia proposta aqui não apresenta oscilações espúrias nos campos
de pressão e velocidade. Devido à formulação de elementos finitos, cada
propriedade $\phi$ é atribuída en cada elemento tetraédrico, assim, uma
transição brusca de propriedades é obtida. Este procedimento de
suavização é necessário para evitar instabilidades numéricas perto da
interface.

Apesar da definição precisa da interface e das transição das
propriedades dos fluidos, mudanças topológicas não são naturalmente
inerentes nesta metodologia, requisitando então um esforço de
implementação na modelagem de coalescência e quebra de interface de
bolhas e gotas. Entretanto, com o uso adequando de modelos geométricos
pode-se modelar a separação e colapso de duas interfaces distintas. Por
exemplo, quando a espessura do filme de líquido que separa duas bolhas
diminui até um determinado valor, as superfícies das bolhas são
conectadas e obtem-se a coalescência das bolhas. Apesar das mudanças
topológicas ocorrerem, os aspectos físicos do problema não são
solucionados. É fato que o problema físico de coalescencia de bolhas e
quebra de interface ainda é tema de pesquisas atuais.

Uma formulação baseada em elementos finitos pode ser encontrada para a
força de tensão superficial considerando o seguinte esquema: 

\begin{equation}
	\frac{1}{We}\mathbf{M} \fvet 
	= 
	\frac{1}{We} \mathbf{\Sigma} \mathbf{G} H_{\lambda}
	\label{eq:surfDiscrete}
\end{equation}

Na equação acima, $\mathbf{\Sigma}$ representa uma matriz diagonal com
elementos $\sigma \kappa_1, \sigma \kappa_2, \sigma \kappa_3,\cdots,
\sigma \kappa_{NV}$, onde $NV$ é o número total de nós da malha
relativos ao campo de pressão. A matriz $\mathbf{G}$ representa a forma
discreta do operador gradiente $\nabla$ e $H_{\lambda}$ é o função
discreta "Heaviside". Equação~(\ref{eq:surfDiscrete}) pode ser substuída
na forma discreta da equação de conservação de quantidade de movimento e
o termo de tensão superficial pode então ser calculado. 

\section{Atividades Desenvolvidas pelo Bolsista JVT e Bolsista IC}

Pelo beneficiário da bolsa de Atração de Jovens Talentos: Gustavo Rabello
dos Anjos
\begin{itemize}
\item estudo e desenvolvimento de modelo em três dimensões e axisimétrico
para escoamentos multifásicos;
\item estudo de escoamentos multifásicos com ondas capilares em bolhas
submetidas a campo de temperatura variável;
\item realização de experimentos computacionais e laboratoriais de
decomposição de biomassa e produção de gases com sedimentação de
material orgânico; de medição da sedimentação/ressuspensão e consumo de
produtos;
\item validação dos modelos desenvolvidos para sistemas existentes com
utilização de tecnologia de última geração;
\item estudo e desenvolvimento de modelos tridimensionais e axisimétricos
para simulação numérica de estrutura de não-equilíbrio em sistemas
químicos, biológico e ambientais;
\item incorporação das seguintes características ao simulador numérico:
paralelisação dos núcleos de cálculo intensivo em clusters baseados em
processadores de vários núcleos (multicore), através do desenvolvimento
de novos precondicionadores para a aceleração dos métodos iterativos
implantados e validação do modelo proposto através de soluções
analíticas e experimentais (com participação de universidades
internacionais);
\item publicação de resultados em canais de comunicação nacionais e
internacionais renomados.
\end{itemize}

Pelo beneficiário da bolsa de Iniciação Científida: Vinícius Augusto Cinquini Mascarenhas
\begin{itemize}
\item Revisão bibliográfica pertinente ao projeto;
\item familiarização com as técnicas desenvolvidas e implementadas no
código numérico;
\item execução de testes no código numérico;
\item aumento da qualificação profissional do aluno de IC através de
desenvolvimento e pesquisa.
\item publicação de trabalho em congresso nacional/internacional.
\end{itemize}

\section{Resultados Obtidos}

Nesta seção buscamos apresentar algumas validações e resultados obtidos
com o desenvolvimento do código de elementos finitos tridimensionais
para simulação de escoamentos multifásicos. As
Figs.~(\ref{fig:pressure3d}, \ref{fig:oscillating} e \ref{fig:sessile})
representam validações através de comparação com resultados analíticos.
A Fig.~(\ref{fig:sucrose}) é resultado de simulação de bolha de ar
imersa em uma solução de sacarose confinada em um canal vertical, movida
devido a força gravitacional (empuxo). Enquanto que na
Fig.~(\ref{fig:sucroseVel}) mostra-se o perfil de velocidade do centro
de massa da bolha em função do tempo para o mesmo caso. Na
Fig.~(\ref{fig:sin}), mostra-se uma simulação numérica de escoamento
multifásico de uma bolha de ar imersa em uma solução de glicerol em um
canal com formato senoidal. Nele é possível notal que a bolha é
comprimida e alongada devido a passagem em partes mais estreitas do
canal seguinda de passagens com expansão. Futuras investigações serão
realizadas no âmbito do presente projeto CAPES.

 \begin{figure}[h]
 	\begin{center}
 		\includegraphics[angle=0, scale=0.5]{figs/pressure3d.pdf}
 	\end{center}
 	\caption{Pressão capilar de uma gota esférica imersa em outra fluido. 
	O salto de pressão pode ser visto na localização da interface que
	separa os fluidos.}
 	\label{fig:pressure3d} 
 \end{figure}

 \begin{figure}[h!]
 	\begin{center}
 		\includegraphics[angle=0, scale=0.5]{figs/oscillating.pdf}
 	\end{center}
	\caption{Amplitude de oscilação de uma gota. Comparação entre a
	solução numérica através do método de elementos finitos e a solução
	analítica para dois níveis de refinamento de mlaha. O período
	analítico é $0.785$ e a taxa de decaimento é mostrada pelas linhas
	abaixo e acima da curva de oscilação. O período encontrado para a
	malha grossa foi de $0.820$ enquanto que para a malha refinada foi
	de $0.783$.}
 	\label{fig:oscillating} 
 \end{figure}

 \begin{figure}[h!]
 	\begin{center}
 		\includegraphics[angle=0, scale=0.5]{figs/sessileShape.pdf}
 	\end{center}
	\caption{Comparação entre a solução numérica do formato da gota
	obtida e a solução
	analítica correspondente para uma gota apoiada com simetrial axial.
	A solução analítica é encontrada através da equação de capilaridade
	de Young-Laplace.}
 	\label{fig:sessile} 
 \end{figure}


 \begin{figure}[h!]
 	\begin{center}
 		\subfloat[$t=0.00$]
			{\includegraphics[angle=90,scale=0.3]{figs/sucrose-1.pdf}}
 		\hspace{0.7cm}
 		\subfloat[$t=1.09$]
			{\includegraphics[angle=90,scale=0.29]{figs/sucrose-2.pdf}}
 		\hspace{0.7cm}
 		\subfloat[$t=1.78$]
			{\includegraphics[angle=90,scale=0.29]{figs/sucrose-3.pdf}}
 		\hspace{0.7cm}
 		\subfloat[$t=3.05$]
			{\includegraphics[angle=90,scale=0.29]{figs/sucrose-4.pdf}}
 		\hspace{0.7cm}
 		\subfloat[$t=7.41$]
			{\includegraphics[angle=90,scale=0.29]{figs/sucrose-5.pdf}}
 	\end{center}
	\caption{Evolução da forma da bolha com o tempo para uma bolha de ar
	imersa em uma solução de sucarose. (a) Forma inicial da bolha para
	$t=0$. (b-d) Mudança da forma da bolha para solução transiente no
	tempo. (e) Forma da bolha final com $t=7.41$.} 
	\label{fig:sucrose} 
 \end{figure}


 \begin{figure}[h]
 	\begin{center}
 		\includegraphics[angle=0, scale=0.5]{figs/sucrose.pdf}
 	\end{center}
	\caption{Elevação de uma bolha de ar do tipo Taylor imersa em uma
	solução de sucarose. A evolução da velocidade de centro de massa da
	bolha é comparada com a velocidade terminal da bolha encontrada no
	mapa de velocidades de \cite{white1962}}.
   \label{fig:sucroseVel}
 \end{figure}

 \begin{figure}[h]
 	\begin{center}
 		\subfloat[$t=0.00$]
			{\includegraphics[angle=0,scale=0.305]{figs/sin-0.png}}
 		\hspace{0.7cm}
 		\subfloat[$t=1.09$]
			{\includegraphics[angle=0,scale=0.29]{figs/sin-1.png}}
 		\hspace{0.7cm}
 		\subfloat[$t=2.15$]
			{\includegraphics[angle=0,scale=0.29]{figs/sin-2.png}}
 		\hspace{0.7cm}
 		\subfloat[$t=3.71$]
			{\includegraphics[angle=0,scale=0.29]{figs/sin-3.png}}
 		\hspace{0.7cm}
 		\subfloat[$t=4.05$]
			{\includegraphics[angle=0,scale=0.29]{figs/sin-4.png}}
 		\hspace{0.7cm}
 		\subfloat[$t=5.18$]
			{\includegraphics[angle=0,scale=0.29]{figs/sin-5.png}}
 	\end{center}
	\caption{Evolução da forma de uma bolha do tipo Taylor em função do
	tempo para uma bolha de ar imersa em uma solução de glicerol em um
	canal ondulado. A forma do canal é definida por uma função senoidal.
	(a) Forma inicial da bolha para $t=0$. Forma da bolha para solução
	transiente no tempo em (b) $t=1.09$, (c) $t=2.15$, (d) $t=3.71$, (e)
	$t=4.05$, (f) $t=5.18$.} 
	\label{fig:sin} 
 \end{figure}

\section{Produção Científica}
Abaixo estão listados os artigos publicados em canais de comunicação
internacional em que o pesquisador bolsista teve participação direta.

\begin{itemize}
	\item Finite Element Simulation of Fingering in Convective Dissolution in
	      Porous Media Rachel M. Lucena, Norberto Mangiavacchi, José Pontes,
	      Anne De Wit, Gustavo Anjos, CCIS -- 2014
	\item Comparative CFD Simulations of Gas Transport in Slug Flow from
		  Periodic Arrays with Single or Multiple Bubbles G. P.
		  Oliveira, N. Mangiavacchi, G. Anjosa, J. Pontes, J. R. Thome,
		  CCIS -- 2014.
	\item Topological Remeshing and Locally Supported Smoothing for
	      Bubble Coalescence in Two-Phase Flows
		  Gustavo Charles P. de Oliveira, Norberto Mangiavacchi, Gustavo Anjos
		  John R. Thome, COBEM -- 2013
	\item Finite Element Analysis of Pressure-Driven Laminar Flow Inside
	      Periodically Staggered Arrays
		  Gustavo Charles P. de Oliveira, Gustavo R. Anjos, José Pontes,
		  Norberto Mangiavacchi, CONEM -- 2014
	\item Numerical Simulation of a Periodic Array of bubbles in a
	      channel
	      N. Mangiavacchi, G.C.P. Oliveira, G. Anjos and J. R. Thome,
		  PAMAC -- 2013
	\item A Survey of Results Concerning Steady Solutions and the
	      Stability of a Class of Rotating Flows
		  José Pontes, Norberto Mangiavacchi, Gustavo Rabello dos Anjos,
		  Carlos Mendeza, Rachel Manhães de Lucena, Gustavo C. P.
		  Oliveira and Davi V. A. Ferreira, CCIS -- 2014
\end{itemize}

\section{Plano de Trabalho}
O plano de trabalho proposto para o primeiro período de 4 bimestres de
projeto pode ser encontrado na Fig.(\ref{fig:plano}, juntamente com a
etapa final do projeto. Nota-se que o plano de trabalho foi reduzido em
2 bimestres, pois o pesquisador bolsista foi aprovado no concurso
público para professor adjunto do Departamento de Engenharia Mecânica da
Universidade do Rio de Janeiro, tomando posse no dia 14 de julho de
2014. Em decorrência deste fato, o projeto teve seu prazo reduzido para
final de julho de 2015, totalizando 10 bimestres. 

 \begin{figure}[ht!]
 	\begin{center}
 		\includegraphics[angle=0, scale=0.5]{figs/plano.png}
 	\end{center}
 	\caption{Plano de trabalho proposto para o período inicial de
	execução do projeto Bolsa de Atração de Jovens Talentos - CAPES.}
 	\label{fig:plano} 
 \end{figure}

A pesquisa bibliográfica foi realizada com sucesso, onde o bolsista
identificou os pontos mais importantes da modelagem matemática e
implementação numérica para realização de sua pesquisa. A partir do 2o.
bimestre foi iniciado o desenvolvimento do simulador numérico, onde
novas funcionalidades foram incorporadas ao já existente código. É
importante notar que o pesquisador bolsista desenvolveu, além de seu
plano de trabalho inicial, um simulador de escoamentos bidimensional em
linguagem de script PYTHON para uso acadêmico, que servirá de apoio nas
aulas ministradas por ele em cursos de graduação e pós-graduação. Alguns
testes e validações foram realizados com sucesso e os resultados podem
ser encontrados na seção de Resultados Obtidos, bem como algumas
simulações pertinentes ao projeto. O pesquisador bolsista participou
ativamente das publicações relacionadas na seção Produção Científica
usando as funcionalidades incorporadas ao simulador através deste
projeto.

\section{Referências Bibliográficas}
\bibliographystyle{plain}
\bibliography{$HOME/projects/misc/latex/referencias}

\end{document}	

\typeout{ ****************** End of file main.tex ****************** }

